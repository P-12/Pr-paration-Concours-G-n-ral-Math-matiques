\documentclass[10pt,a4paper]{article}
\usepackage[utf8]{inputenc}  
\usepackage[T1]{fontenc}       
\usepackage[francais]{babel}
\usepackage{fullpage}
%\usepackage{euler}
\usepackage{amsmath}
%\usepackage{framed}
\usepackage{amsfonts}
\usepackage{amssymb}
\usepackage{pifont}
\usepackage{mathrsfs}
\usepackage[squaren,Gray]{SIunits}
\usepackage{graphicx}
\usepackage[a4paper]{geometry}
\geometry{hscale=0.86,vscale=0.87,centering}

\pagestyle{empty}

%\newcommand{\Titre}[4]{\noindent \textsc{#1}} \hfill \textbf{\textsc{#2}}\\ #3 \hfill \emph{#4}\\  \hrule\vspace{\baselineskip}}

\newcommand{\Titre}[3]{\begin{center} {\LARGE\textbf{\textsc{#1}}}\\ #2 \hfill \emph{#3} \\  \hrule\vspace{\baselineskip}\end{center}}

\begin{document}


\Titre{Préparation au concours Général \\  Arithmétique - Correction}{Asp J.Buet}{mars 2014}
\thispagestyle{plain}
\pagestyle{plain}


%\title{DM de Mathématiques}
%\date{19 novembre 2013}
%\author{Asp. J.Buet}
%\maketitle{}

\section{Cours}

\begin{itemize}

\item \textbf{Principe des tiroirs}
On jette de manière aléatoire de la peinture blanche et de la peinture noire sur une table de $\unit{2}{\meter}$ par $\unit{3}{\meter}$. 
Montrez que l'on peut forcément trouver deux points, distants d'un mètre, de la même couleur.

Pour cela, on place un triangle équilatéral d'un mètre de côté sur la table. Chacun des sommets est sur une zone coloriée en blanc ou en noir. Il y a trois
sommets et deux couleurs, donc il y a au moins deux sommets de la même couleur. Ils sont distants d'un mètre.

\item \textbf{Théorème des restes chinois}

\textbf{Exemple classique : } Une bande de $17$ pirates possède un trésor constitué de pièces d'or d'égale valeur. Ils projettent de se les partager également
et de donner le reste au cuisinier chinois. Celui-ci recevrait alors $3$ pièces. Mais les pirates se querellent, et six d'entre eux sont tués. Un nouveau 
partage donnerait au cuisinier $4$ pièces. Dans un naufrage ultérieur, seuls le trésor, six pirates et le cuisinier sont sauvés, et le partage donnerait alors
$5$ pièces d'or à ce dernier. Quelle est la fortune minimale que peut espérer le cuisinier s'il décide d'empoisonner le reste des pirates\,?

On doit résoudre le système
$$\left\lbrace
\begin{array}{rcl}
n &\equiv& 3 \pmod{17} \\
n &\equiv& 4 \pmod{11} \\
n &\equiv& 5 \pmod{6}
\end{array}\right.$$
 $17$, $11$ et $6$ sont deux à deux premiers entre eux. D'après le théorème des restes chinois, le système admet une unique solution modulo $17\times 11\times 6=1122$.
On cherche $u_1$ et $v_1$ tels que $u_1\times 11\times 6 + v_1\times 17 = 1$. On applique l'algorithme d'Euclide :
$$\begin{array}{ccl}
66 & = & 3\times 17+15 \\
17 & = & 15\times 1 + 2 \\
15 &=& 2\times 7+1
\end{array}.$$
Donc en remontant l'algorithme, on a :
$$\begin{array}{ccl}
1&=& 15-2\times 7 \\
&=& 15- (17-15)\times 7 \\
&=& (66-3\times 17)-(17-(66-3\times 17))\times 7 \\
&=& 8\times 66 - 37\times 17
\end{array}.$$
On a donc $u_1=8$ et $v_1=-31$.

On cherche $u_2$ et $v_2$ tels que $u_2\times 17 \times 6+11\times v_2$. De la même manière on obtient $u_2=4$ et $v_2=-37$.

On chercher $u_3$ et $v_3$ tels que $u_3\times 17\times 11 + 6\times v_3=1$. De la même manière on obtient $u_3=1$ et $v_3=-31$.

D'après la méthode donnée dans la démonstration du théorème des restes chinois, une solution du système est : $8\times 3 \times 6 \times 11+ 4\times 4 \times
17 \times 6+1\times 5 \times 17\times 11=4151$. Pour avoir la plus petite solution possible, on regarde $4151 \pmod{1122}$ soit $785$.

Le cuisinier peut espérer avoir au moins $785$ pièces d'or.

\end{itemize}

\section{Exercices}

\begin{enumerate}
\bigskip
\item $p-1$, $p$ et $p+1$ sont trois nombres consécutis. Donc il y en a au moins un qui est divisible par $3$ et ce n'est pas $p$. $p$ est impair car c'est un
nombre premier $\geqslant 5$. Donc $p-1$ et $p+1$ sont paires et l'un des deux est un multiple de $3$. De plus $p-1$ et $p+1$ sont deux nombres paires 
consécutifs donc l'un des deux est divisible par $4$. Finalement $(p-1)(p+1)$ est divisible par $2\times 4\times 3$ donc $p^2-1$ est divisible par $24$.
\bigskip
\item On regarde modulo $4$.

$(-1)^n-2^m\equiv 1\pmod{4}$. Pour $m\geqslant 2$, on obtient $(-1)^n\equiv 1 \pmod{4}$. Donc $n$ est pair. On note $n=2k$.
$$
\begin{array}{rcl}
3^{2k}-1&=&2^m\\
(3^k-1)(3^k+1)&=&2^m\\
2\alpha(2\alpha+2)&=&2^m
\end{array}$$

En effet, $3^k-1$ et $3^k+1$ sont deux entiers pairs consécutifs. Et ces deux entiers sont des puissances de $2$. Les deux seuls entiers pairs consécutifs qui
sont des puissances de $2$ sont $2$ et $4$. Donc on a forcément $k=1$. Les seuls couples $(n,m)$ convenables sont donc $(1,1)$ et $(2,3)$.
\bigskip
\item On calcule $(1+1)^n$, $(1+j)^n$ et $(1+j^2)^n$ avec le binôme de Newton ($j$ est le complexe $e^{2i\pi/3}$, $j^2=\bar{j}$, $j^3=1$ et $1+j+j^2=0$).
$$\begin{array}{rcl}
(1+1)^n&=&\sum\limits_{k=0}^n\binom{n}{k} \\
(1+j)^n &=&\sum\limits_{k=0}^n\binom{n}{k}j^k \\
(1+j^2)^n&=&\sum\limits_{k=0}^n\binom{n}{k}j^{2k}\\
(1+1)^n+(1+j)^n+(1+j^2)^n&=&\sum\limits_{k=0}^n\binom{n}{k}(1+j^k+j^{2k})
\end{array}. $$
Calculons $1+j^k+j^{2k}$ selon les valeurs de $k$ modulo 3.

Si $k\equiv 0 \pmod{3}$, $1+j^k+j^{2k}=1+1+1=3$.

Si $k\equiv 1\pmod{3}$, $1+j^k+j^{2k}=1+j+j^2=0$.

Si $k\equiv 2\pmod{3}$, $1+j^k+j^{2k}=1+j^2+j=0$.

Finalement, on a donc :
$$\sum\binom{n}{3k}=\frac{(1+1)^n+(1+j)^n+(1+j^2)^n}{3}$$
\bigskip
\item
Pour avoir le chiffre des dizainese de milliers, on regarde le reste dans la division par $10^5$. Or $10^5=2^5\times 5^5$.
Regardons le reste de la division de $5^{5^{5^{5^{5}}}}$ par $2^5$. Or $\varphi(2^5)=16$ (car $\varphi(p^r)=p^r-p^{r-1}$). Donc il suffit de déterminer le reste
de la division de $5^{5^{5^{5}}}$ par $16$.

Comme $\varphi(16)=8$, il suffit de nouveau de déterminer le reste de la division euclidienne de $5^{5^{5}}$ par $8$. Comme $\varphi(8)=4$, il suffit 
de déterminer le reste de la division euclidienne de $5^5$ par $4$. Et comme $\varphi(4)=2$, en prenant le chemin inverse et en appliquant le théorème d'Euler
, on a : 

$$\begin{array}{c}
5^{2.\varphi(4)+1}\equiv 5^1 \pmod{4} \text{ d'après le théorème d'Euler} \\
5^{5^5}\equiv 5^{k.\varphi(8)+5}\equiv 5^5\equiv 5^{\varphi(8)+1}\equiv 5 \pmod{8} \\
5^{5^{5^{5}}}\equiv 5^5 \equiv 3125 \equiv 5 \pmod{16} \\
5^{5^{5^{5^{5}}}}\equiv 5^5 \pmod{2^5}.
\end{array}$$
Et finalement on a : $5^{5^{5^{5^{5}}}}\equiv 5^5 \pmod{2^5\times 5^5}$. Or $5^5=3125$, ce qui implique que le chiffre des dizaines de milliers du nombre en question
est $0$.
\end{enumerate}
\section{Problème : Concours Général 2012 - les premiers sont en haut, les exposants sont en bas}
Correction de PIERRE CORNILLEAU
\noindent
\begin{enumerate}
\item
\begin{enumerate}
\item On a $2012=4\times 503=2^2\times 503$. Par définition on a donc :
$$\boxed{f(2012)=2^2\times 1^{503}=4.}$$
\item On a d'abord $36^{36}=(4\times 9)^{36}=2^{72}3^{72}$. Donc 
$$\boxed{f(36^{36})=72^5=(8\times 9)^5=2^{15}3^{10}}$$
et donc
$$\boxed{f^2(36^{36})=15^210^3=2^33^25^5}$$
puis finalement
$$\boxed{f^3(36^{36})=3^22^35^5=f^2(36^{36}).}$$
Ainsi, pour tout $n\in\mathbb{N}$, $f^{n+2}(36^{36})=f^n(f^2(36^{36}))=f^n(f^3(36^{36}))=f^{n+3}(36^{36})$ et les termes suivants sont tous égaux.
\end{enumerate}
\item
\begin{enumerate}
\item D'après l'énoncé, on a $f(128)=49\neq 128$ et ainsi, puisque $49=7^2$, $f^2(128)=2^7=128$ et nécessairement $f^2(49)=49$.

On a ainsi, pour tout $n\in\mathbb{N}$, $f^{2(n+1)}(128)=f^{2n}(f^2(128))=f^{2n}(128)$ et la suite $(f^{2n}(128))_{n\in\mathbb{N}}$ est donc constante (égale à $128$).
De même, on montre que $(f^{2n+1}(128))_{n\in\mathbb{N}})$ est constante (égale à $49$) et finalement, pour tout $i\in\mathbb{N}$ ,
$$ \boxed{f^{i+2}(128)=f^i(128) \text{ et } f^{i+1}(128)\neq f^i(128).}$$ 
\item On a toujours, d'après l'énoncé, $f(128)=49$ et $f(49)=128 > f(128)$, donc \underline{$f$ n'est pas croissante}. De même, on a $f(720)=128$ et 
$f(128)=49 < f(720)$ donc \underline{ $f$ n'est pas décroissante}.
\end{enumerate}
\item 
\begin{enumerate}
\item L'équation $f(n)=1$ se réécrit $a_1^{p_1}...a_k^{p_k}=1$ et puisque $a_1, ..., a_k$ sont des entiers non nuls, on a nécessairement $a_1=...=a_k=1$ et 
finalement $n=p_1...p_k$. $k$ étant quelconque, les solutions $f(n)=1$ sont \underline{les produits de nombres premiers deux à deux distincts}.
\item L'équation s'écrit $a_1^{p_1}...a_k^{p_k}=2$ et l'un au moins des entiers $a_i$ est supérieur à $2$. Puisque $p_i\geqslant 2$, le mebre de gauche de 
l'égalité est plus grand que $4$ et \underline{il n'y a pas de solutions à l'équation $f(n)=2$}.
\item L'équation s'écirt $a_1^{p_1}...a_k^{p_k}=4$ ; l'un des entiers $a_i$ est supérieur à $2$ et puisque $p_i\geqslant 2$, on a nécessairement $p_i=2$ et 
$a_j=1$ si $i\neq j$. $k$ étant quelconque, l'ensemble cherché est :
$$\boxed{\{4p_1...p_k\text{ avec }k\in\mathbb{N}^*, p_1,..., p_k \text{ nombres premiers deux à deux distincts et différents de } 2\}.}$$
\end{enumerate}
\item
\begin{enumerate}
\item Si $b=0$ ou $1$, le résultat est acquis. Fixons $a\geqslant 2$ et montrons par récurrence sur $b\in\mathbb{N}^*$ que $ab\leqslant a^b$.
\begin{itemize}
\item L'initialisation a été faite ci-dessus.
\item Si le résultat est acquis pour $b\in\mathbb{N}^*$, on a alors 
$$a(b+1)=ab +a\leqslant a^b +a \leqslant a^b+a^b \leqslant 2a^b \leqslant a^{b+1}$$
puisque $b\geqslant 1$ et $a\geqslant 2$. Ainsi
$$\boxed{ \text{si } a\geqslant 2 \text{ et } b\geqslant 0, \text{  on a bien } ab\leqslant a^b.}$$
\end{itemize}
\item Procédons cette fois-ci par récurrence sur le nombre d'entiers $k\in\mathbb{N}^*$.
\begin{itemize}
\item Si $k=1$, le résultat est vrai d'après la question précédente.
\item Si le résultat est acquis pour $k\in\mathbb{N}^*$, il suffit de traiter le cas où tous les $(b_i)_{1\leqslant i\leqslant k+1}$ sont non nuls (sinon seuls $k$ 
entiers interviennent). Dans ce cas, on a alors 
$$a_1b_1+...+a_kb_k+a_{k+1}b_{k+1}\leqslant a_1^{b_1}...a_k^{b_k}+a_{k+1}^{b_{k+1}}$$
d'après la question précédente et l'hypothèses de récurrence. On note $x$ et $y$ les deux entiers du membre de droite de l'inégalité. Il suffit maintenant de 
montrer que 
$$x+u\leqslant xy \text{ soit encore } 1\leqslant (x-1)(y-1)$$
ce qui est vrai car, les entiers $a_i$ étant supérieurs à $2$, on a $x,y\geqslant 2$.

Ainsi, si $k\in\mathbb{N}^*$, $a_1,...,a_k$ des entiers supérieurs à $2$ et $b_1,...,b_k$ sont des entiers
$$\boxed{a_1b_1+...+a_kb_k\leqslant a_1^{b_1}...a_k^{b_k}}$$

\end{itemize}
\item On reprend les notations que la question $3$ et on décompose les nombres $a_1,...,a_k$ en facteurs premiers. En notant $\mathcal{P}$ les nombres premiers 
(intervenant dans l'une des décompositions des $a_i$), on peut écrire :
$$\forall i\in\{1,...,k\}, a_i=\prod\limits_{p\in \mathcal{P}}p^{b_i(p)}$$
où $b_i(p)$ peut être égal à $0$. On a alors :
$$\begin{array}{rl}
f(n)&=a_1^{p_1}...a_k^{p_k}\\
 &=(\prod\limits_{p\in \mathcal{P}}p^{b_1(p)})^{p_1}...(\prod\limits_{p\in \mathcal{P}}p^{b_k(p)})^{p_k}\\
 &= \prod\limits_{p\in\mathcal{P}}p^{p_1b_1(p)+...+p_kb_k(p)}.
\end{array}$$
Ainsi, $f(f(n))=\prod\limits_{p\in\mathcal{P}}(p_1b_1(p)+...+p_kb_k(p))^p$. On a d'abord, d'après la question précédente, et pour tout $1\leqslant i \leqslant k$,
$1b_1(p)+...+p_kb_k(p)\leqslant p_1^{b_1(p)}...p_k^{b_k(p)}$, et donc $f(f(n))\leqslant p_1^{\sum\limits_{p\in\mathcal{P}}pb_1(p)}...p_k^{\sum\limits_{p\in\mathcal{P}}pb_k(p)}.$

Maintenant, la question précédente nous fournit également, pour tou $1\leqslant i \leqslant k$, $\sum\limits_{p\in\mathcal{P}}pb_i(p)\leqslant\prod
\limits_{p\in\mathcal{P}}p^{b_i(p)}=a_i$, et finalement
$$\boxed{f(f(n))\leqslant p_1^{a_1}...p_k^{a_k}=n}$$
\item D'après la question précédente, les deux suites $(f^{2k}(n))_{k\in\mathbb{N}}$ et $(f^{2k+1}(n))_{k\in\mathbb{N}}$ sont décroissantes. En utilisant la descente 
infinie, ces suites sont constantes à partir d'un certain rang. On a donc :
$$\exists k_1,k_2 \in \mathbb{N} ;  \forall k\geqslant k_1, f^{2k}(n)=f^{2k_1}(n) \text{ et } \forall k\geqslant k_2, f^{2k+1}(n)=f^{2k_2+1}(n).$$
Finalement, pour tout $i\geqslant \max(2k_1,2k_2+1)=r$,
$$\boxed{f^{i+2}(n)=f^i(n).}$$
\end{enumerate}
\item 
\begin{enumerate}
\item Si $a$ est pair, il existe $\alpha \in\mathbb{N}$ tel que \underline{$a=2\alpha+3\beta$ avec $\beta=0$}. Sinon, $a$ est impair et supérieur à $3$, il 
existe donc $k\in\mathbb{N}^*$ tel que $a=2k+1=2(k-1)+3\times 1$ et \underline{$(\alpha,\beta)=(k-1,1)$ convient}.
\item Soit $n=p_1^{a_1}...p_k^{a_k} \in E$. D'après la question précédente, pour tou $1\leqslant i \leqslant k$, il existe $\alpha_i, \beta_i$, tels que 
$$a_i=2\alpha_i+3\beta_i$$ et donc $n=(p_1^{\alpha_1}...p_k^{\alpha_k})^2(p_1^{\beta_1}...p_k^{\beta_k})^3=f(2^{p_1^{\alpha_1}...p_k^{\alpha_k}}3^{p_1^{\beta_1}...p_k^{\beta_k}})$ et on peut choisir
\underline{$m=2^{p_1^{\alpha_1}...p_k^{\alpha_k}}3^{p_1^{\beta_1}...p_k^{\beta_k}}$}.
\item On peut effectuer ce qui a été fait dans la question précédente ou écrir plus simplement $2012^{2012}=(2012^{1006})^2=f(2^{2012^{1006}})$ pour choisir 
\underline{$m=2^{2012^{1006}}$}.
\item Soit $n$ tel qu'existe $m$ vérifiant $f(m)=n$. On a, notant $m=p_1^{a_1}...p_k^{a_k}$ et en reprenant les notations de la question $4(c)$, 
$$n=\prod\limits_{p\in\mathcal{P}}p^{p_1b_1(p)+...+p_kb_k(p)}$$
et il suffit alors de constater que, pour tout $p\in\mathcal{P}$,
\begin{itemize}
\item soit $p_1b_1(p)+...+p_kb_k(p)=0$ et alors $b_1(p)=...=b_k(p)=0$,
\item soit il existe $i$ tel que $b_i(p)\geqslant 1$ et donc, puisque $p_i\geqslant 2$, $p_1b_1(p)+...+p_kb_k(p)\geqslant b_i(p)p_i\geqslant 2$.
\end{itemize}
On a donc bien $n\in E$ et \underline{la réciproque de $(b)$ est vraie}.
\end{enumerate}
\end{enumerate}
\end{document}
