\documentclass[10pt,a4paper]{article}
\usepackage[utf8]{inputenc}  
\usepackage[T1]{fontenc}       
\usepackage[francais]{babel}
\usepackage{fullpage}
%\usepackage{euler}
\usepackage{amsmath}
%\usepackage{framed}
\usepackage{amsfonts}
\usepackage{amssymb}
\usepackage{pifont}
\usepackage{mathrsfs}
\usepackage{graphicx}
\usepackage[a4paper]{geometry}
\usepackage{pstricks-add}
\geometry{hscale=0.86,vscale=0.87,centering}

\pagestyle{empty}

%\newcommand{\Titre}[4]{\noindent \textsc{#1}} \hfill \textbf{\textsc{#2}}\\ #3 \hfill \emph{#4}\\  \hrule\vspace{\baselineskip}}

\newcommand{\Titre}[3]{\begin{center} {\LARGE\textbf{\textsc{#1}}}\\ #2 \hfill \emph{#3} \\  \hrule\vspace{\baselineskip}\end{center}}

\begin{document}


\Titre{Préparation au Concours Général \\ Combinatoire - Correction}{Asp J.Buet}{21 mars 2014}
\thispagestyle{plain}
\pagestyle{plain}


%\title{DM de Mathématiques}
%\date{19 novembre 2013}
%\author{Asp. J.Buet}
%\maketitle{}
\section{Exercices}
\subsection{}
C'est le nombre d'applications de l'ensemble $\{1,2,...n\}$ dans l'ensemble $\{0,1\}$. Il y en a donc $2^n$.
\subsection{}
On numérote les $n$ sommets du polygone dans le sens trigonométrique en commençant par l'un quelconque d'entre eux : $A_1, A_2, ..., A_n$. \`A tout ensemble 
de $4$ sommets $A_i, A_j, A_k, A_l$ $(i<j<k<l)$, correspond un point $M$ où $A_iA_k$ et $A_jA_l$ se recoupent.

Réciproquement, à chaque point d'intersection de deux diagonales situés à l'intérieur du polygone correspond un ensemble de $4$ sommets. Il s'ensuit qu'il 
existe $\binom{n}{4}=\frac{n(n-1)(n-2)(n-3)}{4!}$ points d'intersection des diagonales du polygone situés en son intérieur.
\subsection{}
Soit $E$ l'ensemble des chemins menant de $(0,0)$ à $(m,n)$. Si on désigne par un ``\,$0$\,'' un déplacement horizontal et par un ``\,$1$\,'' un déplacement
vertical, il apparaît que tout élément de $E$ peu être uniquement représenté par une suite finie de longueur $m+n$ et à valeurs dans l'ensemble $\{0,1\}$
comportant exactement $m$ ``\,$0$\,''. Si $F$ désigne l'ensemble de ces suites alors $$|F|=\binom{m+n}{m}=\binom{m+n}{n}$$, et comme $E$ et $F$ sont en 
bijection, il s'ensuit que le nombre de façons d'aller de $(0,0)$ à $(m,n)$ par un chemin où à chaque pas l'une des coordonnées augmente d'une unité est 
$$\boxed{|E|=\binom{m+n}{m}=\binom{m+n}{n}.}$$
\subsection{}
$$\sum\limits_{k=0}^n\binom{n}{k}=\sum\limits_{k=0}^n\binom{n}{k}1^k1^{n-k}$$
On reconnaît le binôme de Newton.
$$\boxed{\sum\limits_{k=0}^n\binom{n}{k}=(1+1)^n=2^n}$$

\noindent
Pour le deuxième calcul, on utilise le fait que si $n,k\geqslant 1$,
$$k\binom{n}{k}=n\binom{n-1}{k-1},$$
ce qui permet de conclure :
$$\sum\limits_{k=0}^nk\binom{n}{k}=\sum\limits_{k=1}^nk\binom{n}{k}=\sum\limits_{k=1}^nn\binom{n-1}{k-1}=n\sum\limits_{k=0}^{n-1}\binom{n-1}{k}=n.2^{n-1}$$.
\subsection{}
On note $S(n,p)$ le nombre de surjections d'un ensemble de cardinal $n$ dans un ensemble de cardinal $p$.
\begin{enumerate}
\item $S(n,n)=n!$ car cela correspond au nombre de permutations de $\{1,...,n\}$.

Pour qu'il y ait une surjection de $\{1,...,p+1\}$ dans $\{1,...,p\}$, cela veut dire qu'il y a $2$ éléments de l'ensemble de départ qui ont la même image, 
les autres ont des images deux à deux distinctes. On choisit c'est deux éléments, il y a $\binom{p+1}{2}$ choix possibles. On choisit cette image commune,
il y a $p$ choix. Puis on envoie les $p-1$ éléments restants dans l'ensemble de départ sur les $p-1$ éléments restants sur l'ensemble d'arrivée. Il y a 
$(p-1)!$ manière de le faire. Au final, il y a donc $p\binom{p+1}{2}(p-1)!=\frac{p(p+1)!}{2}$ surjections.
$$\boxed{S(p+1,p)=\frac{p(p+1)!}{2}}$$
\item On prend un élément quelconque dans l'ensemble d'arrivée. Cet élément a entre $1$ et $n$ antécédents. Supposons qu'il en a $k$, on choisit alors 
$k$ éléments dans l'ensemble de départ qui seront ses antécédents, il y a $\binom{n}{k}$ manières de les choisir. Puis on envoie les $n-k$ éléments restants
sur les $p-1$ éléments restants de manière surjective, il y a $S(n-k,p-1)$ manières de le faire. On somme sur tous les $k$ possibles et on obtient :
$$\boxed{S(n,p)=\sum\limits_{k=1}^n\binom{k}{n}S(n-k,p-1)}$$
\end{enumerate}
\subsection{}
Soit $n\in\mathbb{N}$, il y a deux cas, soit on a une partie qui contient l'entier $n$, soit elle ne le contient pas.

Si elle ne contient pas l'entier $n$ alors il y en a $A_{n-1}$.

Si elle contient l'entier $n$, alors elle ne peut pas contenir l'entier $n-1$, donc elle ne contient que des entiers dans $\{0,...,n-2\}$ en plus de l'entier
 $n$ et que des entiers non-consécutifs de $\{0,...,n-2\}$. Donc il y en a $A_{n-2}$ tels.

Finalement $A_n=A_{n-1}+A_{n-2}$.

\end{document}
