\documentclass[10pt,a4paper]{article}
\usepackage[latin1]{inputenc}  
\usepackage[T1]{fontenc}       
\usepackage{fullpage}
%\usepackage{euler}
\usepackage{amsmath}
%\usepackage{framed}
\usepackage{amsfonts}
\usepackage{amssymb}
\usepackage{pifont}
\usepackage{mathrsfs}
\usepackage{graphicx}
\usepackage[a4paper]{geometry}
\usepackage{pstricks-add}
\geometry{hscale=0.86,vscale=0.87,centering}

\pagestyle{empty}

%\newcommand{\Titre}[4]{\noindent \textsc{#1}} \hfill \textbf{\textsc{#2}}\\ #3 \hfill \emph{#4}\\  \hrule\vspace{\baselineskip}}

\newcommand{\Titre}[3]{\begin{center} {\LARGE\textbf{\textsc{#1}}}\\ #2 \hfill \emph{#3} \\  \hrule\vspace{\baselineskip}\end{center}}

\usepackage[frenchb]{babel}

\begin{document}

\Titre{Pr�paration au concours G�n�ral \\ Combinatoire}{Asp J. Buet}{17 mars 2014}
%\Titre{Pr�paration au Concours G�n�ral \\ Combinatoire}{Asp J.Buet}{17 mars 2014}
\thispagestyle{plain}
\pagestyle{plain}


%\title{DM de Math�matiques}
%\date{19 novembre 2013}
%\author{Asp. J.Buet}
%\maketitle{}
\section{Exercices}
\subsection{}
Quel est le nombre de suites finies de longueur $n$ et � valeurs dans $\{0,1\}$?
\subsection{}
Quel est le nombre de points d'intersection des diagonales d'un polyn�me convexe � $n$ sommets situ�s � l'int�rieur de ce polygone sachant que trois
quelconques de ses diagonales ne sont pas concourantes en un m�me point\,?
\subsection{}
Soit une grille rectangulaire $\mathbb{N}\times\mathbb{N}$.
\begin{center}
\begin{pspicture*}(-3.1,-0.1)(2.1,3.1)
\psgrid[subgriddiv=0,gridlabels=0,gridcolor=lightgray](0,0)(-3.1,-0.1)(2.1,3.1)
\psset{xunit=1.0cm,yunit=1.0cm,algebraic=true,dotstyle=o,dotsize=3pt 0,linewidth=0.8pt,arrowsize=3pt 2,arrowinset=0.25}
\psline[linewidth=1.6pt](-3,0)(-1,0)
\psline[linewidth=1.6pt](-1,0)(-1,1)
\psline[linewidth=1.6pt](-1,1)(1,1)
\psline[linewidth=1.6pt](1,1)(1,3)
\psline[linewidth=1.6pt](1,3)(2,3)
\begin{scriptsize}
\psdots[dotstyle=*,linecolor=blue](-3,0)
\rput[bl](-2.9,0.12){$(0,0)$}
\psdots[dotstyle=*,linecolor=blue](-3,3)
\rput[bl](-2.9,2.7){$(0,n)$}
\psdots[dotstyle=*,linecolor=blue](2,3)
\rput[bl](1.2,2.7){$(m,n)$}
\psdots[dotstyle=*,linecolor=blue](2,0)
\rput[bl](1.2,0.12){$(m,0)$}
\end{scriptsize}
\end{pspicture*}
\end{center}

Quel est le nombre de fa�ons d'aller de $(0,0)$ �  $(m,n)$ sans retourner en arri�re (c'est-�-dire par un chemin o� �  chaque pas l'une des coordonn�es augmente
d'une unit�)\,?
\subsection{}
Calculer $\sum\limits_{k=0}^n\binom{n}{k}$.

Prouver que, pour $n\geqslant 1$,  $\sum\limits_{k=0}^nk\binom{n}{k}=n.2^{n-1}$.
\subsection{}
On note $S(n,p)$ le nombre de surjections d'un ensemble de cardinal $n$ dans un ensemble de cardinal $p$.
\begin{enumerate}
\item Calculer $S(n,n)$ et $S(p+1,p)$.
\item Montrer que : $S(n,p)=\sum\limits_{k=1}^n\binom{n}{k}S(n-k,p-1)$.
\end{enumerate}
\subsection{}
Soit $A_n$ le nombre de parties de $\{1,...,n\}$ qui ne contiennent pas deux entiers cons�cutifs. 

Montrer que $A_n=A_{n-1}+A_{n-2}$.
\subsection{}
On dispose de $b$ boules blanches et $n$ boules noires, que l'on r�partit entre deux urnes de fa�on qu'aucune d'elles ne soit vide. On note $s$ le nombre de
boules dans la premi�re, et $r$ celui de ces boules qui sont au hasard. Le but de l'exercice est de d�terminer les r�partitions rendant maximale la 
probabilit� $p$ de tirer une boule blanche.
\begin{enumerate}
\item Exprimez $p$ en fonction de $b$, $n$, $r$ et $s$ ;
\item Dans cette question, on fixe la valeur de $s$. Comment choisir $r$ pour augmenter $p$\,?
\item R�solvez l'exercice.
\end{enumerate}

\end{document}
