\documentclass[10pt,a4paper]{article}
\usepackage[utf8]{inputenc}  
\usepackage[T1]{fontenc}       
\usepackage[francais]{babel}
\usepackage{fullpage}
%\usepackage{euler}
\usepackage{amsmath}
%\usepackage{framed}
\usepackage{amsfonts}
\usepackage{amssymb}
\usepackage{pifont}
\usepackage{mathrsfs}
\usepackage[squaren,Gray]{SIunits}
\usepackage{graphicx}
\usepackage[a4paper]{geometry}
\geometry{hscale=0.86,vscale=0.87,centering}

\pagestyle{empty}

%\newcommand{\Titre}[4]{\noindent \textsc{#1}} \hfill \textbf{\textsc{#2}}\\ #3 \hfill \emph{#4}\\  \hrule\vspace{\baselineskip}}

\newcommand{\Titre}[3]{\begin{center} {\LARGE\textbf{\textsc{#1}}}\\ #2 \hfill \emph{#3} \\  \hrule\vspace{\baselineskip}\end{center}}

\begin{document}


\Titre{Préparation au concours Général \\  Arithmétique}{Asp J.Buet}{mars 2014}
\thispagestyle{plain}
\pagestyle{plain}


%\title{DM de Mathématiques}
%\date{19 novembre 2013}
%\author{Asp. J.Buet}
%\maketitle{}

\section{Cours}

\begin{itemize}
\item Connaître le cours d'arithmétique de spécialité mathématique.
\bigskip
\item Connaître le principe de récurrence et la récurrence forte.
\bigskip
\item \textbf{Principe des tiroirs} Si $n$ balles sont placées dans $k$ tiroirs, au moins un tiroir contiendra $\lceil n/k \rceil$ balles ou plus.

\bigskip
Par exemple, si vous avez des chaussettes vertes et des chaussettes bleues, indiscernables au touché, rangées en vrac. Votre chambre est dans le noir et vous
 voulez avoir deux chaussettes de la même couleur pour vous habiller. Il vous suffit pour cela de prendre trois chaussettes, vous en aurez forcément deux de
la même couleur.

\bigskip
Autre exemple : on jette de manière aléatoire de la peinture blanche et de la peinture noire sur une table de $\unit{2}{\meter}$ par $\unit{3}{\meter}$. 
Montrez que l'on peut forcément trouver deux points, distants d'un mètre, de la même couleur.
\bigskip
\item \textbf{La descente infinie} Tout partie non vide de $\mathbb{N}$ admet un plus petit élément. On dit que $\mathbb{N}$ est bien ordonné. Une conséquence
de ce principe équivalent au principe de récurrence est la méthode de descente infinie découverte par Fermat.

Soit une suite de propositions $P_0,P_1,...,P_n,...$ Si pour tout $k\in\mathbb{N}$ tel que $P_k$ est fausse, il existe $m\in\mathbb{N}$ tel que $P_m$ soit 
fausse et $m<k$, alors toutes les propositions $P_0, P_1, ..., P_n, ...$ sont vraies.
\bigskip
\item \textbf{Indicatrice d'Euler}

Soit $n$ un entier naturel $>1$. On appelle indicatrice d'Euler de $n$ que l'on note $\varphi(n)$, le nombre d'entiers naturels non nuls $\leqslant n$ et premiers avec $n$.

\begin{itemize}
\item Si $p$ est un nombre premier alors $\varphi(p)=p-1$
\item La fonction $\varphi$ est multiplicartive, i.e., pour tout couple d'entiers naturels $m$ et $n$ premiers entre eux on a : 
$$\varphi(mn)=\varphi(m)\varphi(n).$$
\end{itemize}
\bigskip
\item \textbf{Théorème de Wilson}

$p$ divise $(p-1)!+1$ si, et seulement si, $p$ est premier. (Caractérisation des nombres premiers).
\bigskip
\item \textbf{Théorème d'Euler}

Si $a$ est premier avec $n$ alors :
$$a^{\varphi(n)} \equiv 1 \pmod n$$
\bigskip
\item \textbf{Théorème des restes chinois}

On considère des nombres naturels $m_1, m_2, ...,m_k, $ premiers entre eux deux à deux, et des entiers $b_1, b_2, ..., b_k$. Le système de congruences : 
$$\left\lbrace
\begin{array}{c}
x \equiv b_1  \pmod {m_1}\\
x \equiv b_2  \pmod {m_2} \\
... \\
x \equiv b_k \pmod {m_k}
\end{array}\right.$$
admet une unique solution modulo $M=m_1m_2...m_k$.
\\

Voir le poly à part pour les démonstrations et une méthode de résolution du dernier système.
\\

\textbf{Exemple classique : } Une bande de $17$ pirates possède un trésor constitué de pièces d'or d'égale valeur. Ils projettent de se les partager également
et de donner le reste au cuisinier chinois. Celui-ci recevrait alors $3$ pièces. Mais les pirates se querellent, et six d'entre eux sont tués. Un nouveau 
partage donnerait au cuisinier $4$ pièces. Dans un naufrage ultérieur, seuls le trésor, six pirates et le cuisinier sont sauvés, et le partage donnerait alors
$5$ pièces d'or à ce dernier. Quelle est la fortune minimale que peut espérer le cuisinier s'il décide d'empoisonner le reste des pirates\,?
\bigskip
\item Vous pourrez voir en regardant les annales que l'année du concours revient souvent dans le problème. Je vous conseille donc de regarder dès maintenant
quelques propriétés de l'entier $2014$ (diviseurs, diviseurs premiers, nombre de diviseurs, $\varphi(2014)$...).
\end{itemize}

\section{Exercices}

\begin{enumerate}
\bigskip
\item Soit $p\geqslant 5$ un nombre premier. Montrer que $24$ divise $p^2-1$.
\bigskip
\item
Trouver tous les couples $(n,m)\in\mathbb{N}^2$ tels que $3^n-2^m=1$
\bigskip
\item
Calculer :
$$\binom{n}{0}+\binom{n}{3}+\binom{n}{6}+...$$
\bigskip
\item
Quel est le chiffre des dizaines de milliers du nombre $5^{5^{5^{5^{5}}}}$\,?
\end{enumerate}
\section{Problème : Concours Général 2012 - les premiers sont en haut, les exposants sont en bas}
\noindent
Pour tout entier $n\geqslant 2$, on dispose de la décomposition en facteurs premiers
$$n=p_1^{a_1}p_2^{a_2}...p_k^{a_k}$$
où les nombres premiers disctincts $p_1,p_2,...p_k$ sont les diviseurs premiers de $n$, et les exposants $a_1,a_2,...a_k$ sont des entiers strictement positifs
. On pose alors 
$$f(n)=a_1^{p_1}a_2^{p_2}...a_k^{p_k}.$$
Par exemple, si $n=720=2^43^25^1$, on a $f(n)=4^22^31^5=128$.
\\
En posant de plus $f(1)=1$, on obtient une fonction $f$ de $\mathbb{N}^*$ dans $\mathbb{N}^*$.
\\
Enfin, pour $n\in\mathbb{N}^*$, on définit $f^i(n)$ par récurrence sur $i\in\mathbb{N}$, de façon que $f^0(n)=n$ et
$$\text{pour tout $i\in\mathbb{N}$, } f^{i+1}(n)=f(f^i(n)).$$
Par exemple : $f^0(720)=720, f^1(720)=f(720)=128, f^2(720)=f(128)=49.$
\\
Le but de ce problème est d'étudier le comportement de la fonction $f$ et des suites $(f^i(n))_{i\in\mathbb{N}}$ pour $n$ fixé.
\begin{enumerate}
\item
\begin{enumerate}
\item Calculer $f(2012)$.
\item Déterminer les nombres $f^i(36^{36})$ pour $0\leqslant i \leqslant 3$. Que peut-on dire des suivants\,?
\end{enumerate}
\item
\begin{enumerate}
\item Donner un exemple d'entier $n\geqslant 1$ tel que, pour tout entier naturel $i$, on ait
$$f^{i+2}(n)=f^i(n) \text{ et } f^{i+1}(n)\neq f^i(n).$$
\item Montrer que la fonction $f$ n'est ni croissante ni décroissante.
\end{enumerate}
\item Résoudre dans $\mathbb{N}^*$ :
\begin{enumerate}
\item l'équation $f(n)=1$ ;
\item l'équation $f(n)=2$ ;
\item l'équation $f(n)=4$.
\end{enumerate}
\item
\begin{enumerate}
\item Pour tous entiers $a\geqslant 2$ et $b\geqslant 0$, montrer que $ab\leqslant a^b$.
\item Soit $k\in\mathbb{N}^*$ et soit $a_1,...,a_k,b_1,...,b_k$ des entiers tels que $a_i\geqslant 2$ et $b_i\geqslant 0$ pour tout $i$.

Montrer que
$$a_1b_1+a_2b_2+...+a_3b_3\leqslant a_1^{b_1}a_2^{b_2}...a_k^{b_k}.$$
\item Pour tout $n\in\mathbb{N}^*$, montrer que $f(f(n))\leqslant n$.
\item Soit $n\in\mathbb{N}^*$. Montrer qu'il existe un entier naturel $r$ tel que, pour tout entier $i\geqslant r$, on ait $f^{i+2}(n)=f^i(n)$.
\end{enumerate}
\item Soit $E$ l'ensemble des entiers $n\geqslant 2$ n'ayant que des exposants strictements supérieurs à $1$ dans leur décomposition en facteurs premiers.
\begin{enumerate}
\item Pour tout entier $a\geqslant 2$, montrer qu'il existe des entiers naturels $\alpha$ et $\beta$ tels que 
$$a=2\alpha+3\beta.$$
\item En déduire que si $n$ appartient à $E$, alors il existe un élément $m$ de $E$ tel que $f(m)=n$.
\item Donner un élément $m$ de $E$ tel que $f(m)=2012^{2012}$.
\item Que peut-on dire de la réciproque de (b)\,?
\end{enumerate}
\end{enumerate}
\end{document}
